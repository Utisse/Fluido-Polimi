\lecture{13}{lun 27 mar 2023 14:30}{Sistemi di riferimento non cartesiani, Hagen-Poiseuille, Taylor-Covette}

\section{Sistemi di Riferimento non cartesiani}
Se cambio l'origine della terna non ho problemi nel definire la nuova terna.
In tutti gli altri casi invece si presentano diversi problemi.
Per esempio se passo da  una terna in 3D composta da vettori ortonormali $ \vec{i},\vec{j},\vec{k} $ è molto semplice cambiare l'origine della terna e ridefinire i punti dello spazio in base alla nuova terna.
Se ho coordinate circolari il discorso si complica ma non troppo.\\
Applico il cambio di coordinate a Navier-Stokes
\begin{gather*}
  \underline{\nabla}\cdot \underline{u} = 0 \rightarrow  \frac{1}{r}\pdev{}{r}\left( ru_r \right)  + \frac{1}{r}\pdev{u_\theta }{\theta }+\pdev{u_z}{z}=0\\
  \underline{u}\cdot \underline{\nabla} \underline{u} \rightarrow \text{ sarebbero 11 termini!!}
\end{gather*}
Diventa evidente che il discorso si complica (le equazioni sono comunque leggibili dal Kunda).\\

\subsection{Flusso di Hagen-Poiseiulle}
Consideriamo il flusso di Poiseuille in 3D, chiamato anche flusso di Hagen-Poiseuille.
Le nostre ipotesi sono:
\begin{itemize}
\item Problema stazionario: $ \pdev{}{t}=0 $ 
\item Direzione $ z $ omogenea: $ \pdev{u}{t}=\pdev{v}{t} =0 $
\item $ \pdev{P}{z} \exists $ ed è costante (se non c'è gradiente non c'è movimento)
\item Problema assialsimmetrico: $ \pdev{}{\theta }=0 \rightarrow u_\theta =\operatorname{cost} $
\item Fluido incomprimibile
\item No forze di volume (forza peso)  
\end{itemize}
Da queste ipotesi posso dire che 
\[
  u = u_z\left( r_k\right) \vec{k}
\]
Aggiungo la \emph{no slip condition}
\[
  u\left( R\right) =0
\]
Analizzo le varie equazioni
\begin{gather*}
  -\mu \nabla^2u_z + \pdev{P}{z}=0\\
  -\mu \frac{1}{r}\pdev{}{r}\left( r\pdev{u_z}{r}\right) +G_P=0\\
  -\mu \frac{1}{r}\frac{d }{d r}\left( r\frac{d u_z}{dr }\right) +G_P\\
  \mu \frac{1}{r}\left( r u^{\prime}\right)^{\prime}=G_P
\end{gather*}
Servono due condizioni al contorno:
\[
  \left( ru^{\prime}\right) ^{\prime}=\frac{G_P}{\mu }r
\]
\begin{equation*}
  \begin{cases}
    u\left( 0\right) = \text{ Non deve divergere a centro canale} \\
    u\left( R\right) =0 \text{ Velocità nulla a parete} 
  \end{cases}
\end{equation*}
Integro due volte e trovo 
\[
  u\left( r\right)  = \frac{G_P}{4\mu }r^2+A\ln r+B
\]
Aggiungendo a questa equazione le condizioni al contorno trovo 
\begin{equation}
  u\left( r\right) = \frac{G_P}{4\mu }\left( r^2-R^2\right) 
\end{equation}
Analizzando il caso pratico e fisico trovo che 
\begin{gather*}
  u_{\text{max}} = - \frac{G_P}{4\mu }R^2\\
  Q = -\frac{\pi}{8}G \frac{R^4}{u} \rightarrow \overline{U} = \frac{Q}{\pi R^2}=-\frac{GR^2}{8\mu }=\frac{1}{2}u_{\text{max}}\\
  G = - \frac{{8 \mu \overline{U}}}{R^2}\rightarrow G^* = \frac{G}{\rho \overline{U}^2}\left( 2R\right) = -\frac{16}{Re}=-16\frac{\mu}{\rho \overline{U}R}
\end{gather*}
Nella seguente tabella si illustra le differenze tra il flusso piano e quello cilindrico
\begin{table}[H]
  \centering
  \Large
  \begin{tabular}[H]{c|c|c|c}
    & $ u_\text{max}  $ & $ \overline{U} $ & $ G^* $\\ \hline
    Piano & $ -\frac{Gh^2}{8\mu } $ & $ \frac{2}{3}u_\text{max}  $ & $-\frac{12}{Re}$\\
    Cilindrico & $ -\frac{GR^2}{4\mu } $ & $ \frac{1}{2}u_\text{max}  $ & $-\frac{16}{Re}$\\
                                              
  \end{tabular}
\end{table}

\section{Problema di Taylor-Covette}
\begin{figure}[H]
  \centering
  \incfig{copp_reg}{}
\end{figure}

\begin{gather*}
  \pdev{}{t} = 0 \qquad \pdev{}{\theta }=0\\
  u = u_\theta \left( r\right) \underline{e_\theta}\\
  \underline{u}\left( R_1\right)  = \Omega_1 R_1\underline{e_\theta}\\
  \underline{u}\left( R_2\right)  = \Omega_2 R_2\underline{e_\theta}\\
  \underline{u}\cdot \underline{\nabla} \underline{u} =-\frac{1}{r}u_\theta ^2 \underline{e_r}\\
  \nabla^2 \underline{u} = \left[ \frac{1}{r}\pdev{}{r}\left( r\pdev{u_\theta }{r}-\frac{1}{r^2}u_\theta \right] e_z
  \end{gather*}
Le tre equazioni di navier-stokes diventano
  \begin{equation*}
    \begin{cases}
      -\frac{1}{r}u_\theta ^2=-\frac{1}{\rho }\pdev{P}{r}\\
      -\frac{1}{r}\pdev{}{r}\left( r\pdev{u_\theta }{r}\right) -\frac{1}{r^2}u_\theta =0 \rightarrow \underbrace{\frac{d }{d r}{\left[ \frac{1}{r}\frac{d }{d r}\left( ru_\theta \right) \right] }}_{\operatorname{cost}}=0\\
      0 =0
    \end{cases}
\end{equation*}
\[
  \frac{1}{r}\frac{d }{d r}\left( r u_\theta \right) =A \rightarrow d\left( ru_\theta \right)  = Ardr \rightarrow r u_\theta  = \frac{1}{2}Ar^2+B \rightarrow u_\theta =\frac{1}{2}Ar+ \frac{B}{r}
\]
Ma ancora non so quali sono le due costanti $ A $ e $ B $.
Le trovo con le condizioni al contorno che ho scritto prima.
Cioè:
\begin{equation*}
  \begin{cases}
    u_\theta \left( R_1\right) =\Omega _1R_1=\frac{1}{2}A R_1 + \frac{B}{R_1}\\
    u_\theta \left( R_2\right) =\Omega _2R_1=\frac{1}{2}A R_2 + \frac{B}{R_2}
  \end{cases}
\end{equation*}
Sostituisco $ B $:
\[
  \Omega _2 R_2^2 - \Omega_1 R_1^2 = \frac{1}{2}A\left( R_2^2-R_1^2\right) 
\]
E trovo:
\[
  A = 2 \frac{{\Omega_2 R_2^2 - \Omega _1R_1^2}}{R_2^2-R_1^2} \qquad\qquad B = \frac{{\left( \Omega_1 - \Omega _2\right)  R_1^2R_2^2}}{R_2^2-R_1^2} 
\]
\[
  u_\theta  = \frac{1}{2}Ar + \frac{B}{r}
\]

\subsection{Casi Particolari}
\begin{itemize}
\item Per $ R_1 \to 0$ , $  \Omega _1 \to 0 \implies  u_\theta \left( r\right)  = \Omega r $ 
\item Per $ R_2 \to \infty $ , $  \Omega _2 \to 0 \implies  u_\theta \left( r\right)  = \Omega_1 R_1^2 \frac{1}{r} $ 
\end{itemize}
\\
\hline
\vspace{1ex}
%%% Local Variables:
%%% mode: latex
%%% TeX-master: "master"
%%% End:
