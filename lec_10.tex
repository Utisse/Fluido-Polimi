\lecture{10}{ven 17 mar 2023 14:30}{Bernulli}
Date le precendenti supposizioni riguardandi un fluido:
\[
\mu \sim 0  \quad \mu_v \sim0 \quad \underline{g}=-\underline{\nabla}\phi \quad \rho =\rho \left( P\right) \iff P=P\left( \rho \right) 
\]
e cioè, fluido non viscoso (1 e 2), ipotesi di forze conservative di volume (3) e flusso barotropico (4), posso dedurre che la densità e la pressione non dipendono dalla temperatura. Le seguenti affermazioni sono valide per il nostro caso particolare del flusso barotropico:
\begin{itemize}
\item $ \rho  $ = costante ovunque
\item $ T $ = costante
\item $ \delta  $ costante
\end{itemize}
Defnisco dunque un potenziale barotroppico
\[
dF = \frac{dP}{\rho }
\]
Se è un differenziale esatto allora
\[
F\left( P\right)  = \displaystyle{\int_{P_0}^{P}} \frac{dP'}{\rho \left( P'\right) }
\]
Se $ P = \operatorname{cost}  $ allora l' integrale è immediato $ F = \frac{{P-P_0}}{\rho } $. Per calcolare il gradiente del potenziale:
\[
\pdev{F}{x_i}=\pdev{F}{P}\cdot \pdev{P}{x_i} =\frac{1}{\rho }\pdev{P}{x_i}
\]
che in forma vettoriale diventa
\[
\underline{\nabla}F=\frac{1}{\rho }\underline{\nabla}P
\]
\hline
\vspace{2ex}
Se posso trascurare gli attriti vale l' equazione di Crocco \cref{eq:crocco}
\begin{gather*}
  \pdev{\underline{u}_j}{t}- \left( \underline{u}\times \underline{\omega}\right) _j + \pdev{}{x_j}\left( \frac{1}{2}u^2\right) +\frac{1}{\rho }\pdev{P}{x_j}+\pdev{\phi }{x_j}=0\\
  \pdev{\underline{u}}{t}-\underline{u}\times \underline{\omega}+\underline{\nabla}\left( \frac{1}{2}u^2\right)  + \underline{\nabla}\displaystyle{\int_{P_0}^{P}} \frac{dP'}{\rho }+ \underline{\nabla}\phi =0\\
  \pdev{\underline{u}}{t} - \underline{u}\times \underline{\omega} + \underline{\nabla}\left\underbrace{( \frac{1}{2}u^2+\displaystyle{\int_{P_0}^{P}}\frac{dP^\prime}{\rho }+\phi \right)}_{\text{Trinomio B}} = 0 \to\\
  \to \pdev{\underline{u}}{t}-\underline{u}\times \underline{\omega}+\underline{\nabla}B=0
\end{gather*}
Questa equazione costituisce la base per le tre versioni della prima forma del teorema di bernulli:
\begin{enumerate}
\item Campo stazionario, $ \pdev{\underline{u}}{t}=0 $ e irrotazionale $ \underline{\omega}=0 $ 
  \[
    \rightarrow \underline{\nabla}B=0 \to \frac{1}{2}u^2+\displaystyle{\int_{P_0}^{P}} \frac{{dP^\prime}}{\rho }+\phi = \operatorname{cost}
  \]
  Nell' ipotesi di $ \rho  $ costante, cioè di flusso incomprimibile 
  \[
    \frac{1}{2}u^2+\frac{{P-P_0}}{\rho }+\phi =\operatorname{cost} 
  \]
  Essendo la pressione definita a meno di una costante:
  \[
    \frac{1}{2}u^2 + \frac{P}{\rho }+\phi =\operatorname{cost} +\frac{P_0}{\rho }=\operatorname{cost}
  \]
  \begin{equation}
    \rightarrow P + \frac{1}{2}\rho u^2+\rho \phi =\operatorname{cost}
  \end{equation}
\item Campo rotazionale $ \underline{\omega} \neq 0$ ma stazionario $  \pdev{\underline{u}}{t} = 0 $
  \[
\rightarrow \underline{u}\times \underline{\omega} = \underline{\nabla}B \to \underline{u}\times \underline{\omega} \parallelsum \underline{\nabla}B 
\]
Consideriamo superifici in cui $ \underline{\nabla}B = \operatorname{cost}  $ 
\begin{equation}
  \frac{1}{2}q^2+\displaystyle{\int_{}^{}}\frac{dP}{\rho }+gz = \operatorname{cost} 
\end{equation}
Costante ma solo su line di corrente e vorticose (su tutto solo se $ \underline{\omega}=0 $ 
\item Campo irrotazionale $ \underline{\omega} = 0 $ e non stazionario $ \pdev{\underline{u}}{t} \neq 0 $
  Dato che il campo è irrotazionale allora:
  \[
    \underline{\nabla}\times  \underline{u} = 0 \rightarrow \underline{u}=\underline{\nabla}\varphi 
  \] 
  \begin{gather*}
    \pdev{\underline{u}}{t}+\underline{\nabla}B = 0 \\
    \pdev{}{t}\underline{\nabla} \varphi +\underline{\nabla} B =0 \\
    \underline{\nabla} \left( \pdev{}{t}\varphi +B \right) =0
  \end{gather*}
  \begin{equation}
    \rightarrow \pdev{\varphi }{t}+B = \operatorname{cost} 
  \end{equation}
\end{enumerate}

\section{Corrente ideale}
Parliamo di corrente ideale quando il flusso è stazionario $ \pdev{\underline{u}}{t}=0 $, il flusso di calore è nullo $ \underline{q}=0 $ e $ \delta =\operatorname{cost}  $ 
\begin{equation*}
  \frac{D }{Dt}\left( \frac{1}{2}u^2+e\right) =u_ig_i+\frac{1}{\rho }\pdev{}{x_j}\left( u_i \tau_{ij} \right) +\cancel{\frac{1}{\rho }\pdev{}{x_i}q_i}
\end{equation*}
Prendo in esame il termine $ \frac{1}{\rho }\pdev{}{x_j}\left( u_i\tau_{ij} \right)  $:
\begin{gather*}
  \frac{1}{\rho }\pdev{}{x_j}\left( u_i\tau_{ij} \right) =\frac{1}{\rho }\left( -u_jP\delta_{ij} \right) +\cancel{\frac{1}{\rho }\pdev{}{x_j}\left( u_i \sigma_{ij}\right) }=\\
  =-\frac{1}{\rho }\underline{\nabla} \cdot \left( \ddot{u}P = -\frac{1}{\rho }\left( \underline{u}\cdot \underline{\nabla} P+P\underline{\nabla} \cdot \underline{u}\right)=\\
    = -\frac{1}{\rho }\frac{D P}{Dt}+\frac{P}{\rho ^2}\frac{D P}{Dt}=-\frac{D }{Dt}\left( \frac{P}{\rho }\right) 
\end{gather*}
Prendo in esame il termine $ u_ig_i $
\begin{gather*}
  u_ig_i = \underline{u}\cdot \underline{g} = -\underline{u}\cdot \underline{\nabla} \phi =-\frac{D \phi }{Dt} \\
  \frac{D \phi }{Dt}=\pdev{\phi }{t}+u\cdot \underline{\nabla} \phi \\
  \rightarrow \frac{D }{Dt}\left( \frac{1}{2}u^2+e\right) =-\frac{D \phi }{Dt}\frac{D }{Dt}\left( \frac{P}{\rho }\right) \rightarrow \\
  \rightarrow \frac{D }{Dt} \underbrace{{\left( \frac{1}{2}u^2+e+\phi +\frac{P}{\rho }\right) }}_{H} = 0
\end{gather*}
  In generale le 4 componenti di $ H $ possono variare ma $ H $ è costante lungo una traiettoria, non cambia.
\subsection{Tubo di Venturi}
\begin{figure}[ht]
    \centering
    \incfig{venturi_tube}{}
    \caption{Tubo di Venturi}
    \label{fig:venturi_tube}
\end{figure}
In un tubo di venturi, nel canale stretto il fluido accelera e la pressione decresce quindi il foro centrale aspira. Così funzionano i carburatori. Questo è derivabile da $ H $
\[
  H = P + \frac{1}{2}\rho u^2+\rho \phi =\operatorname{cost} 
\] 
In un profilo alare, nel punto di ristagno ($ u=0 $) la pressione è massima. Poi sul dorso dell'ala accelera velocemente quindi la pressione diminuisce. Sul ventre l'accelerazione è minore quindi la pressione è maggiore la differenza di pressione genera portanza
\subsection{Tubo di Pitot}
\begin{figure}[ht]
    \centering
    \incfig{tubo_di_pitot}{}
    \caption{Tubo di Pitot}
    \label{fig:tubo_di_pitot}
\end{figure}
Per la legge di stevino 
\begin{gather*}
  P_1=P_{atm}+\rho g h_1\\
  P_2=P_{atm}+\rho g h_2\\
  P_2-P_1=\rho g \left( h_2-h_1\right) =\rho g \Delta h
\end{gather*}
Per la legge di Bernulli
\begin{gather*}
  \frac{1}{2}\rho u^2_1 + P_1 = \cancel{\frac{1}{2}\rho u^2_2} + P_2\\
  P_2 -P_1 = \frac{1}{2}\rho u^2_1 = \rho g \Delta h\\
  \rightarrow u_1 = \sqrt{\frac{{2\Delta P}}{\rho }}= \sqrt{{\rho g\Delta h}} 
\end{gather*}
Posso misurare $ u_1 $ con $ \Delta h $. Il tubo di Pitot però può misurare la velocità a Mach $ \approx 0 $. L' errore $ >1\% $ per Mach $ >0.3 $. Per questo motivo $ 0.3 $ è valore soglia di comprimibilità

%%% Local Variables:
%%% mode: latex
%%% TeX-master: "master"
%%% End:
