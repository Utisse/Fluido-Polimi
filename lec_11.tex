\lecture{11}{mar 20 mar 2023 14:30}{Similitudine Dinamica}

\section{Similitudine Dinamica}
Devo semplicemente guardare il numero di Reynolds:
\begin{equation}
  \label{eq:numero_reynolds}
  Re =\frac{\rho uD}{\mu }= \frac{uD}{\nu }
\end{equation}

\subsection{Teorema di Buckingham}
Date $ n $ variabili e $ k $ grandezze fondamentali il teorema di Buckingham afferma che il problma h $ n-k $ `gruppi` (anche detti parametri) adimensionali.\\
Prendiamo in considerazione le equazioni di Navier-Stokes. \\
Abbiamo $ n=9 $ variabili
\[
  x_i,t,u_i,P,g_i
\]
e $ k=6 $ grandezze
\[
  L,\frac{1}{T}=\Omega ,U,\rho,\mu,\abs{g} 
\]
Dunque per il teorema di Buckingham ho 3 parametri. Definisco 
\[
  x_i^*=\frac{x_i}{L}, t^*=\Omega t, u_i^*=\frac{u_i}{U},P^*=\frac{P}{\rho U^2},g_i^*=\frac{g_i}{\abs{g}}
\]
Posso far diventare adimensionale Navier-Stokes, trovando
\[
  t = \frac{t^*}{\Omega},x_i=x_i^*L, \dots 
\]
\[
  U\Omega \pdev{u^*}{t^*}+\frac{U^2}{L}\underline{u}^*\cdot \underline{\nabla}^*\underline{u}^* = -\frac{U}{L}\underline{\nabla} ^*P^*+g \underline{g}^*+ \frac{\mu}{\rho }\frac{U}{L^2}\underline{\nabla}^{*2} \underline{u}^*
\]
Riscrivo per mettere in evidenza i coefficienti adimensionali
\[
  \underbrace{{\left( \frac{\Omega L}{U}\right)}}_{St} + \underline{u}^*\cdot \underline{\nabla} ^*\underline{u}^*=-\underline{\nabla} ^*P^*+\underbrace{{\left( \frac{gL}{U}\right)}}_{\frac{1}{Fn^2}}\underline{g}^* + \underbrace{{\left( \frac{\mu}{\rho UL}\right) }}_{\frac{1}{Re}}\nabla ^{*2}\underline{u}^*
\]
Dunque abbiamo trovato i tre parametri richiesti dal teorema di Buckingham:
\begin{itemize}
  \large
\item \textbf{Strouhal}
  \begin{equation}
    St = \frac{{\Omega L}}{U} = \frac{\text{Accelerazione instazionaria}}{\text{Accelerazione convettiva}}
  \end{equation}
\item \textbf{Froude}
  \begin{equation}
    Fr = \frac{U}{\sqrt{gL}}= \frac{\text{Forza di inerzia}}{\text{Forza peso}}
  \end{equation}
\item \textbf{Reynolds}
  \begin{equation}
    Re = \frac{{\rho LU}}{\mu } = \frac{\text{Forze di inerzia}}{\text{Forze Viscose}}
  \end{equation}
\end{itemize}

%%% Local Variables:
%%% mode: latex
%%% TeX-master: "master"
%%% End:
