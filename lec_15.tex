\lecture{15}{ven 03 apr 2023 14:30}{Aerodinamica classica, Circuitazione}
Lungo un profilo alare abbiamo un profilo della velocità di un fluido che va come mostrato in figura:
% TODO: Aggiungi figura e sistema la roba di inkscape, magari
Lo strato intorno al profilo lungo il quale la velocità del fluido è nulla si chiama \emph{strato limite}.\\ 
All'interno dello strato limite avremo 
\[
  \underline{\omega} = \underline{\nabla} \times \underline{u} \neq 0
\]
Fuori dallo strato limite invece:
\[
  \underline{\omega} = \underline{\nabla} \times \underline{u} = 0
\]
Inoltre ricordiamo che $  \underline{\omega} = \underline{\nabla}\times \underline{\omega} $ e che la divergenza del rotore è nulla sempre: $ \underline{\nabla}\cdot \underline{\omega}= \underline{\nabla}\cdot \left( \underline{\nabla}\times \underline{u} \right)$\[
	\underline{\nabla}\times \dermat{\underline{u}}{t} = \underline{\nabla}\times(\cancel{-\frac{1}{\rho}\underline{\nabla}P} + \cancel{\cancel{g}} + \nu \nabla ^2 \underline{u}
\]
\[
	\underline{\nabla}\times \dermat{\underline{u}}{t} = \underline{\nabla}\times \left( \nu \nabla ^2 \underline{u}
 \right)\]
 Altri equazioni rilevanti sono:
 \begin{itemize}
	 \item $ \underline{g} = \nabla \phi  \rightarrow \underline{\nabla}\times g = \underline{\nabla}\times \left(  \underline{\nabla} \phi \right) = 0$ 
	 \item $ \frac{ \left( \underline{\nabla}\times \underline{\nabla}P \right)}{-\rho} = 0 $
	 \item $ \underline{\nabla}\times \dermat{\underline{u}}{t} = \underline{\nabla}\times \pdev{\underline{u}}{t} + \underline{\nabla}\times \left( \underline{u}\cdot \underline{\nabla}\underline{u} \right) =\pdev{}{t}\left( \underline{\nabla}\times \underline{u} \right) + \underline{\nabla}\times \left(  \underline{u}\cdot \underline{\nabla}\underline{u} \right)$
		 \[
			 \rightarrow \underline{u}\cdot \underline{\nabla} u = u_j \pdev{u_i}{x_j} = u_j \pdev{u_i}{x_j} = u_j \left( \pdev{u_i}{x_j}-\pdev{u_j}{x_i} \right) + {u_j}\pdev{{u_j}}{{x_i}}=
		 \]
		 \[
			 = {u_j}\varepsilon_{kjl} {\omega}_{k} + \pdev{}{{x_i}}\left( \frac{1}{2}{u_j}{u_j} \right) - \underline{{\omega}}{\times}\underline{u}+\underline{\nabla}\left( \frac{1}{2} u ^2 \right) 
		 \]
 \end{itemize}
 Dunque troviamo che:
 \[
	 \rightarrow \underline{\nabla}{\times}\left( \underline{u}{\cdot}\underline{\nabla} \right) = \underline{\nabla}{\times}\left( \underline{{\omega}}{\times}\underline{u} \right) + \cancel{\underline{\nabla}{\times}\underline{\nabla}\left( \frac{1}{2} u ^2 \right)} = \underline{\nabla}{\times}\left( \underline{ {\omega}}{\times}\underline{u} \right)
 \]
 \[
   \underline{\nabla}{\times}{\nabla}^2 \underline{u} = {\nabla}^2 \underline{\nabla}{\times}\underline{u}= {\nabla}^2 {\omega}
 \]
 \[
	 \pdev{\underline{{\omega}}}{t} + \underline{\nabla}{\times} \left( \underline{{\omega}} {\times}\underline{u} \right) = {\nu}{\nabla}^2 \underline{u}
 \]
 Sfruttando l'identità vettoriale $ \underline{\nabla}{\times}\left( {\underline{\omega}}{\times} \underline{u} \right) = \underline{u}{\cdot}\underline{\nabla}{\omega}-{\underline{\omega}}{\cdot}\underline{\nabla}\underline{u} $
 \[
	 \pdev{\underline{{\omega}}}{t} + \underline{u}{\cdot}\underline{\nabla}{\omega}- \underline{{\omega}}{\cdot}\underline{\nabla}\underline{u}= {\nu}{\nabla}^2 \underline{u}
 \]
 Raggiungo così l'equazione
 \begin{equation}
   \dermat{\underline{{\omega}}}{t} = \underline{{\omega}} {\cdot} \underline{\nabla}\underline{u}+ {\nu}\underline{\nabla}^2 \underline{u}
 \end{equation}
 Il primo termine di questa equazione inidica il \emph{vortex stretching}, cioè quanto si allargano i vortici mentre il secondo termine è il termine \emph{viscoso}.

\section{Circolazione}
La circolazione $ {\Gamma} $ è la circuitazione della velocità.
\[
  {\Gamma} ={ \integral{C}{}{\underline{u}}{\underline{x}}  }  = { \integral{A}{}{\underline{\nabla}{\times}\underline{u}}{A}  } = \integral{A}{}{\underline{{\omega}}}{A}
\]
\[
  {\Gamma} = \integral{C}{}{\underline{n}}{\underline{x}} = \integral{C}{}{{u_i}}{{x_i}}
\]
\[
  \dermat{{\Gamma}}{t} = \dermat{}{t} \integral{C}{}{{u_i}}{{x_i}} = \integral{C}{}{\dermat{{u_i}}{t}}{{x_i}} + \integral{C}{}{{u_i \dermat{}{t}}}{{x_i}}
\]
Il secondo termine della somma, $ \integral{C}{}{{u_i \dermat{}{t}}}{{x_i}} $ si sviluppa nel seguente modo:
\[
	\integral{C}{}{{u_i}}{{u_i}} = \integral{C}{}{}{\left( \frac{1}{2} {u_i}{u_i} \right)} = \integral{C}{}{}{\left( \frac{1}{2} u ^2 \right)} = \left.  \frac{1}{2} u ^2\right|_{P_1}^{P_1} = 0 
\]
Il secondo termine invece $ \integral{C}{}{\dermat{{u_i}}{t}}{{x_i}} $:
\[
	\integral{C}{}{\dermat{{u_i}}{t}}{{x_i}} = \integral{C}{}{-\frac{1}{{\rho}}\pdev{P}{{x_i}}+g_i+\frac{1}{{\rho}}\pdev{\sigma_{ij}}{{x_j}}}{{x_i}}
\]
continuando con lo sviluppo:
\[
	\cancel{\integral{C}{}{\left( - \frac{1}{{\rho}} \right)}{P}} + \cancel{\integral{C}{}{}{{\phi}}} + \integral{C}{}{\frac{1}{{\rho}} \pdev{\sigma_{ij}}{{x_j}}}{{x_i}} 
\]
\subsection{Teorema di Kelvin}
L'ultimo termine restante dipende dal coefficiente di viscosità.
Se la viscosità è trascurabile allora raggiungo finalmente il \emph{teorema di Kelvin}.
Per un flusso, trascurando l'attrito o comunque considerandolo nullo:
\begin{equation}
	\label{eq:teo_kelvin}
	\dermat{{\Gamma}}{t} = 0
\end{equation}
Questo teorema spiega i vortici in scia di un velivolo. 
Nel caso l'attrito non sia nullo allora:
\[
	\underline{\nabla}{\cdot}\underline{u} = 0 \quad {\nabla}{\times}u = 
	\begin{cases}
	  0\\
		\costante
	\end{cases}
\]
\[
  \pdev{\sigma_{ij}}{t} = {\mu} {\nabla}^2 \underline{u}= -{\mu}\underline{\nabla}{\times}\underline{{\omega}}
\]
Il termine $ \underline{\nabla}{\times}\underline{{\omega}} $ è nullo se $ \underline{{\omega}} = 0 $ oppure se $ \underline{{\omega}} = \costante $.

\subsection{Linea Vorticosa}
La definizione di linea vorticosa è:
\[
  \frac{\diff x}{{\omega}_{x}} = \frac{\diff y}{{\omega}_{y}} = \frac{\diff z}{{\omega}_{z}}
\]
\[
  {\omega}_{{\theta}} = \frac{{\omega}_{r}}{2} = {\Omega} r
\]
Abbiamo una \emph{Rotazione Rigida} se:
\[
  {\omega} = \underline{\nabla}{\times}\underline{u} \neq0
\]
Oppure un \emph{vortice irrotazionale} se:
\[
  u _{{\theta}} = \frac{{\Gamma}}{2 {\pi}r} \quad u_r=0
\]
\[
  {\omega} = \underline{\nabla}{\times}\underline{u}=0
\]
Il centro della rotazione (dove $ u_{{\theta}}= \infty $) si dice \emph{singolarità}.
\subsubsection{Tubo vorticoso}
Analogo del tubo di flusso per la velocità:
\[
  {\Gamma}= \oint \underline{u}\space \diff \underline{x}
\]
