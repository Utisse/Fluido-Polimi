\lecture{20}{lun 08 mag 2023 14:30}{Strato Limite}
Quando troviamo la corrente intorno alla superficie di un corpo, in aerodinamica classica stiamo facendo una semplificazione: sulla superficie consideriamo esclusivamente la componente tangenziale alla superficie.
In generale è una buona approssimazione ma non vedremo mai:
\begin{itemize}
\item Sforzo a parete $ \tau _w $ manca la resistenza
\item Non abbiamo mai separazione.
\end{itemize}
Se il corpo è tozzo (non aerodinamico) allora le linee di corrente si separano dal corpo.
Dietro il corpo tozzo si formano due grandi vortici dati dalla ricircolazione, abbiamo anche una contro corrente dietro il corpo.
I grandi vortici in realtà si rompono in tanti più piccoli e generano turbolenza.
Nel \textit{leading edge} c'è alta pressione mentre dietro non si riforma quella pressione a causa della separazione.
Questa resistenza viene chiamata resistenza di \emph{scia} o di \emph{forma}.\\

Entrando più nel dettaglio:
% \begin{figure}[ht]
%     \centering
%     \incfig{velocità_parete}{}
%     \caption{velocità_parete}
%     \label{fig:velocità_parete}
% \end{figure}
Unendo tutti i punti in cui la velocità va a zero troviamo quello che chiamiamo \emph{strato limite} $ \delta  $ .
Fino ad ora abbiamo considerato $ \delta \simeq 0 $. Risulta necessario per una rappresentazione corretta considerare lo strato limite. Un caso in cui è possibile considerare $ \delta =0 $ è nel caso di flusso potenziale, cioè quando il flusso non si separa mai e dunque lo strato limite segue la superficie.\\

\section{Strato Limite}
Ipotesi:
\[
  \begin{cases}
    Re\gg 1\\
    \frac{\delta}{L}\ll 1
  \end{cases}
\]
\begin{figure}[H]
    \centering
    \incfig{strato_lim}{}
    \caption{}
    \label{fig:strato_lim}
\end{figure}
Flusso stazionario $ \pdev{}{t}=0  $, 2D, $ \rho =\operatorname{cost}  $ , $ \mu =\operatorname{cost}  $
\begin{equation*}
  \large
  \begin{cases}
  \underline{\nabla}\cdot \underline{u}=0\\
  \cancel{\pdev{u}{t}}+ u \pdev{u}{x} + v \pdev{u}{y}=-\frac{1}{\rho }\pdev{P}{x}+\nu \left( \pdev{^2u}{x^2}+\pdev{^2u}{y^2}\right) \\
  \cancel{\pdev{y}{t}}+ u \pdev{y}{x} + v \pdev{y}{y}=-\frac{1}{\rho }\pdev{P}{x}+\ny \left( \pdev{^2y}{x^2}+\pdev{^2y}{y^2}\right) \\
  \end{cases}
\end{equation*}
Dobbiamo stimare delle derivate:

\[
  u \pdev{u}{x}\approx  \frac{U^2_\infty}{L}\qquad v\pdev{u}{y}\approx \frac{1}{\delta }
\]
Dato che 
\[
  \pdev{u}{x}+\pdev{v}{y}= 0 \rightarrow \pdev{u}{x}\approx \pdev{v}{y}
\]
Allora le due derivate devono essere comparabili dunque

\[
  \pdev{v}{y} \simeq \frac{U_\infty}{L} \rightarrow  v \approx U_\infty \frac{\delta}{L}
\]
E dunque
\[
  \rightarrow v \pdev{u}{y}\approx U_\infty \frac{\delta}{L} \frac{U_\infty}{\delta } \approx \frac{U_\infty^2}{L}
\]
Termini diffusivi:
\[
  \nu \pdev{^2u}{x^2}\approx \nu \frac{U_\infty}{L}\quad; \quad \nu \pdev{^2u}{y^2}\simeq \nu \frac{U_\infty}{\delta ^2} \gg  \nu \frac{U_\infty}{L^2}
\]
Nello strato limite c'è un equilibrio tra effetti convettivi del flusso e diffusivi dati dall' interazione con la parete. Hp: L'ordine dei termini convettivi dovrà essere pari a quello dei diffusivi. E dunque
\[
  \frac{U_\infty^2}{L}\approx \nu \frac{U_\infty}{\delta ^2} 
\]
Allora: 
\[
  \rightarrow \delta ^2 \approx  \frac{\nu L}{U_\infty  } \quad \delta \approx \sqrt{\frac{{\nu L}}{U_\infty }}
\]
trovo
\[
  \frac{\delta}{L} \approx \sqrt{\frac{{\nu L}}{U_\infty L^2}} = \sqrt{\frac{{\nu}}{U_\infty L}} = \frac{1}{\sqrt{Re}}
\]
\hline
\vspace{1ex}
\[
\]

\begin{gather*}
  x^* = \frac{x}{L}| y^* = \frac{y}{\delta }=\frac{y}{L}\sqrt{Re}| u^* = \frac{u}{U_\infty }|\\
  v^* = \frac{v}{U_\infty }\frac{L}{\delta }=\frac{v}{U_\infty }\sqrt{Re}| P^*=\frac{{P-P_\infty }}{\rho U_\infty ^2}
\end{gather*}
Da questo momento in avanti $ U_\infty =U $
\[
  \pdev{u^*}{x^*}\frac{U}{L}+ \pdev{v^*}{y^*}\frac{U}{\cancel{\sqrt{Re}}}\frac{\cancel{\sqrt{Re}}}{L}\rightarrow 
\]
\[
  \rightarrow  \pdev{u^*}{x^*}+ \pdev{v^*}{v^*}=0
\]

\[
  u^* \pdev{u^*}{x^*}\frac{U^2}{L}+v^*\pdev{u^*}{y^*}\frac{U^2}{\sqrt{Re}}\frac{\sqrt{Re}}{L} = -\frac{1}{\rho} \pdev{P^*}{x^*}{\frac{{\rho U^2}}{L} + \nu  \pdev{^2u^*}{x^{*2}} \frac{U}{L^2}Re
\]
\[
  u^* \pdev{u^*}{x^*}+v^*\pdev{u^*}{y^*}= -\pdev{P^*}{x^*} + \underbrace{\frac{\nu}{UL}}_{\frac{1}{Re}} \pdev{^2u^*}{x^{*2}} + \cancel{\frac{\nu}{UL}}\cancel{Re} \pdev{^2u^*}{y^{*2}}
\]
\begin{equation}
  u^* \pdev{u^*}{x^*}+v^*\pdev{u^*}{y}= - \pdev{P^*}{x^*}+\underbrace{{\left( \frac{1}{Re}\pdev{^2u^*}{x^{*2}}\right)}}_{Re \gg 1 \rightarrow  \approx 0}  + \pdev{^2u^*}{y^{*2}}
\end{equation}
Ragionamento analogo per la seconda equazione di navier stokes (rispetto a $ y $ )
\[
  u^*\pdev{v^*}{x^*}\frac{U^2}{\sqrt{Re}L} + v^* \pdev{v^*}{y^*}\frac{U^2}{\sqrt{Re}L} = -\pdev{P^*}{y^*}\frac{U^2}{L}\sqrt{Re} + \nu \pdev{^2v^*}{y^{*2}}\frac{U}{\sqrt{Re}L^2}+ \nu \pdev{^2v^*}{y^{*2}} \frac{{U \sqrt{Re}}}{L^2}
\]
Moltiplico tutto per $ \frac{L}{U^2 \sqrt{Re}} $ 
\[
  \frac{1}{Re} \left(  u^* \pdev{y^*}{x^*}+v^*\pdev{v^*}{y^*}\right)  = - \pdev{P^*}{y^*}+ \underbrace{{\nu \frac{U}{\sqrt{Re}L^2} \frac{L}{U^2 \sqrt{Re}}}}_{\frac{1}{Re^2}}\pdev{^2v^*}{x^{*2}} + \underbrace{{\nu  \frac{U \sqrt{Re}}{L^2} \frac{L}{U^2 \sqrt{Re}}}}_{\frac{1}{Re}} \pdev{^2v^*}{y^{*2}}
\]
\begin{equation}
  \frac{1}{Re} \left( \text{termini convettivi}\right) = - \pdev{P^*}{y^*}+\frac{1}{Re}\left( \text{termini dissipativi}\right) 
\end{equation}


\begin{equation}
  \large
  \begin{cases}
    \pdev{u^+}{x^*}+\pdev{v^*}{y^*}=0\\
    u^*\pdev{u^*}{x^*}+v^*\pdev{u^*}{y^*}=-\pdev{P^*}{x^*}+\pdev{^2u^*}{y^{*2}}\\
    \pdev{P^*}{y^*}=0
  \end{cases}
\end{equation}
Ritornando ai valori dimensionali
\begin{equation}
  \large
  \begin{cases}
    \pdev{u}{x}+\pdev{v}{y}=0\\
    u\pdev{u}{x}+v\pdev{u}{y}=-\frac{1}{\rho }\pdev{P}{x}+\pdev{^2u}{y^2}\\
    \pdev{P}{y}=0 \rightarrow P = P\left( x\right) 
  \end{cases}
\end{equation}
Vale bernulli 
\begin{gather*}
  P+ \frac{1}{2}\rho U_e^2=\operatorname{cost} =P_\infty +\frac{1}{2}\rho U_\infty ^2 \qquad U_e = U_e \left( x\right) \\
  \frac{dP}{dx} = \frac{d}{dx}\left( -\frac{1}{2}\rho U_e^2\right)  = -\rho U_e \frac{dU_e}{dx}
\end{gather*}

\subsection{Equazioni di Prandlt}

Ora posso ottenere il sistema delle equazioni di Prandlt per strato limite bidimensionale 2D

\begin{equation}
	\label{eq:prandlt}
  \begin{cases}
    \pdev{u}{x}+\pdev{v}{y}=0\\
    u \pdev{u}{x}+v\pdev{u}{y}= U_e \frac{dU_e}{dx} + \nu \pdev{^2u}{y^2}
  \end{cases}
\end{equation}
Abbiamo due equazioni in due incognite: $ u $ e $ v $.\\
Condizioni al contorno:

\begin{gather*} 
  u\left( x,0\right) =v\left( x,0\right) =0 \\
  \underline{u}\left( x,y\rightarrow \infty \right) \rightarrow U_e\\
  \underline{u}\left( 0,y\right)  = \underline{u}_{\text{in}}\left( y\right) 
\end{gather*}
Per la $ x $ e per la $ y $ sto usando due modelli fisici diversi. La seconda equazione mi garantisce che concordino e cioè che per $ y\to\infty, \, \underline{u}\to U_e$.
La terza condizione mi impone di avere una velocità input nota, perchè non si è ancora sviluppata. In generale fuori dallo strato limite vale laplace $ \nabla^2\phi  $.\\
\hline
\vspace{2ex}
Nel modello potenziale l'attrito non si vede.
Precipita di scatto a parete mantendo il proprio valore.
Nella realtà questa soluzione vale solo fuori dalla parete.
Dunque la corrente esterna è quella a parete perchè ora c'è lo strato limite cioè un $ \delta  $ che prima non consideravamo.

%%% Local Variables:
%%% mode: latex
%%% TeX-master: "master"
%%% End:
