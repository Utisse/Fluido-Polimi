\lecture{6}{mar 07 mar 2023 14:30}{Conservazione della quantità di moto}
Ricapitolando abbiamo trovato due equazioni per la conservazione della massa.\\
Una in forma macroscopica:
\[
\pdev{\rho }{t}+ \nabla \left( \rho \underline{u}\right) =0
\]
e una in forma integrale, macroscopica:
\[
\displaystyle{\int_{V}^{}}\left( \pdev{\rho }{t}+\nabla \left( \rho \underline{u}\right) \right) dv = 0
\]
Analizzando la forma integrale troviamo che è composta da due addendi:
\[
\displaystyle{\int_{V}^{}}\pdev{\rho }{t}dv + \displaystyle{\int_{V}^{}}\nabla \cdot \left( \rho \underline{u}\right) dv
\]
Il primo addendo rapprensenta la variazione di massa all'interno del volume, il secondo il flusso entrante o uscente dal volume. Se il flusso è comprimibile posso accumulare massa altrimenti la massa rimane costante $ \dot{m}_i = \dot{m}_u $. Se nel volume indrucessi un esplosivo avrei bisogno di un termine sorgente e dunque l'equazione non sarebbe più nulla ma uguale ad un certo termine sorgente $ S $:
\[
\displaystyle{\int_{V}^{}}\left( \pdev{\rho }{t} + \underline{\nabla} \left( \rho \underline{u}\right) \right) dv = S
\]



%%% Local Variables:
%%% mode: latex
%%% TeX-master: "master"
%%% End:
