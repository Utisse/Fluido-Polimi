\fontsize{20}{18.5}\selectfont
%------------------------------------------------------------------------------
%	REQUIRED PACKAGES AND  CONFIGURATIONS
%------------------------------------------------------------------------------
%%%
%%%   DRACULA THEME
%%%
%\usepackage{/home/lorenzo/latex_dracula/draculatheme}
%PACKAGES FOR TITLES
\usepackage{titlesec}
\usepackage{color}
\usepackage{titling}

% PACKAGES FOR LANGUAGE AND FONT
\usepackage[utf8]{inputenc}
\usepackage[italian]{babel}
\usepackage[T1]{fontenc} % Font encoding

%TIKZ 
\usepackage{tikz}

% TIKZ PACKAGES
\usetikzlibrary{calc,patterns,angles,quotes}
\usepackage{pgfplots}
\pgfplotsset{compat = newest,width=6cm}
\usetikzlibrary{patterns.meta,decorations.pathmorphing}

% PACKAGES FOR IMAGES
\usepackage{graphicx}
\usepackage{float}
\usepackage{wrapfig}
\usepackage{import}
\usepackage{cancel}
\usepackage{xifthen}
\usepackage{pdfpages}
\usepackage{xcolor}
\usepackage{transparent}
\newcommand{\incfig}[2]{%
    \ifthenelse{\isempty{#2}}{%
        \def\svgwidth{.75\columnwidth}
    }{%
        \def\svgwidth{#2\columnwidth}
    }%
    \import{./figures/}{#1.pdf_tex}
}
% PACKAGES FOR MATH
\usepackage{amsmath}
%\usepackage{siunitx}
% PACKAGES FOR REFERENCES & BIBLIOGRAPHY
\usepackage[colorlinks=true,linkcolor=blue!40!black,anchorcolor=black,citecolor=black,filecolor=black,menucolor=black,runcolor=black,urlcolor=black]{hyperref} % Adds clickable links at references
\usepackage[italian]{cleveref}
\pdfsuppresswarningpagegroup=1
%-------------------------------------------------------------------------
%	NEW COMMANDS DEFINED
%-------------------------------------------------------------------------
% EXAMPLES OF NEW COMMANDS -> here you see how to define new commands
\newcommand{\e}[1]{\times 10^{#1}}  % Powers of 10 notation
\newcommand{\mathbbm}[1]{\text{\usefont{U}{bbm}{m}{n}#1}} % From mathbbm.sty
\newcommand{\pdev}[2]{\frac{\partial#1}{\partial#2}}
\newcommand{\integral}[4]{\int_{#1}^{#2} #3 \,\,\diff #4}
\newcommand{\dermat}[2]{\frac{\operatorname{D}#1}{\operatorname{D}#2}}
\newcommand{\abs}[1]{|#1|}
\newcommand{\person}[1]{\textsf{#1}}
\newcommand{\R}{\mathbbm{R}}
\newcommand{\costante}{\operatorname{cost}}
%\newcommand{\ln}{\operatorname{ln}}
\newcommand{\diff}{\operatorname{d}}
\newcommand{\vvline}{||}
\newcommand{\prodc}{\displaystyle\prod}
\newcommand{\parallelsum}{\mathbin{/\mkern-5mu/}}
\newcommand{\twomini}[2]{\begin{figure}[H]
                \centering
		\begin{minipage}{.5\linewidth}
			#1
		\end{minipage}
		\begin{minipage}{.5\linewidth}
			#2
		\end{minipage}
\end{figure}}
\newcommand{\twominisw}[4]{\begin{figure}[H]
                \centering
		\begin{minipage}{#3\linewidth}
			#1
		\end{minipage}
		\begin{minipage}{#4\linewidth}
			#2
		\end{minipage}
              \end{figure}}
            \newcommand{\threemini}[3]{\begin{figure}[H]
                \centering
                \begin{minipage}{.3\linewidth}
			#1
		\end{minipage}
		\begin{minipage}{.3\linewidth}
			#2
		\end{minipage}
		\begin{minipage}{.3\linewidth}
			#3
		\end{minipage}
\end{figure}}
%----------------------------------------------------------------------------
%	COLOURS 
%----------------------------------------------------------------------------
\definecolor{darkorange}{RGB}{229, 112, 0}
\definecolor{darkpurple}{RGB}{119, 25, 170}
\usepackage{xifthen}
\makeatother
\def\@lecture{}%
\newcommand{\lecture}[3]{
    \ifthenelse{\isempty{#3}}{%
        \def\@lecture{Lezione #1}%
    }{%
        \def\@lecture{Lezione #1: #3}%
    }%
    \subsection*{\@lecture}
    \marginpar{\small\textsf{\mbox{#2}}}
}

\usepackage{fancyhdr}
\pagestyle{fancy}

% LE: left even
% RO: right odd
% CE, CO: center even, center odd
% My name for when I print my lecture notes to use for an open book exam.
% \fancyhead[LE,RO]{Lorenzo Pasqui}

\fancyhead[RO,LE]{\@lecture} % Right odd,  Left even
\fancyhead[RE,LO]{Lorenzo Pasqui}          % Right even, Left odd

\fancyfoot[RO,LE]{}  % Right odd,  Left even
\fancyfoot[RE,LO]{}          % Right even, Left odd
\fancyfoot[C]{\thepage}     % Center

\makeatother

% redefinition of \maketitle with a logo ==============  
\makeatletter
\newcommand{\logo}[1]{\gdef\@logo{#1}}%
% \def\maketitle{
% \begin{center}
% \setlength\baselineskip{8ex}
% \setlength\parskip{4em plus 1fil minus 3em}
% \includegraphics[width=\textwidth,
% height=.6\textwidth,keepaspectratio]{\@logo}\par
% {\Huge\sffamily\bfseries \@title}\par
% {\Large\scshape \@author}\par
% \end{center}
% \newpage}
% \makeatother
%  end of redefinition  ===============================  \pdfsuppresswarningpagegroup=1
\logo{~/PoliLogo.svg}
\author{Lorenzo Pasqui}
\newcommand{\myimage}{\includegraphics[width=0.5\textwidth]{PoliLogo}}
\newcommand{\mytitlepage}[1]{%
    \begin{center}
        \myimage
        \vspace{1cm} % Aggiungi spazio tra l'immagine e il titolo
        \LARGE #1
    \end{center}
}

%%% Local Variables:
%%% mode: latex
%%% TeX-master: t
%%% End:
