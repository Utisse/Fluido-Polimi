\lecture{7}{gio 09 mar 2023 14:30}{Equazioni di Navier Stokes e di Eulero}
Definisco la \emph{viscosità volumetrica} come:
\[
\mu_v = \lambda +\frac{2}{3}\mu 
\]
dunque il tensore $ \tau _{ii} $ diventa:

\begin{align*}
  \tau _{ii} = -3P_S+\mu_v \underline{\nabla}\cdot \underline{u}\\
  P_S = - \frac{1}{3}\tau _{ii} + \mu_v \underline{\nabla}\cdot \underline{u}\\
  P_S = \overline{P} + \mu_v \underline{\nabla}\cdot \underline{u} 
\end{align*}
Dunque quali di queste due pressioni ($ P_S $, pressione termodinamica e $ \overline{P} $, pressione meccanica) è quella `vera'?
$ P_S = \overline{P} $ solo se $ \mu_v \underline{\nabla}\cdot \underline{u} = 0 $\\
Dunque sono uguali solo per $ \mu_v =0 $ e cioè solo se il fluido è incomprimibile.
Questa viene chiamata \emph{assunzione} \emph{di} \emph{Stokes}.\\
Da ora in poi la notazione sarà la seguente $ P = P_S = \overline{P} $.\\
Scriviamo ora la relazione costitutiva per i fluidi newtoniani:

\begin{equation}
  \tau _{ii} = -P_S\delta_{ij} +2\mu \left( S_{ij} -\frac{1}{3}S_{kk} \delta_{ij} \right) + \mu _v S_{kk} \delta _{ij}
\end{equation}
Se il fluido è incomprimibile (ricordiamo $ \mu _v = 0 $) allora $ \tau _{ij} $ dipende linearmente. Inoltre $ \underline{\nabla}\cdot \underline{u} =S_{kk} = 0 $. Dunque:
\[
\tau _{ij} = -P_S \delta _{ij} +2\mu S_{ij}
\]
Ora sostituisco nella legge di bilancio:
\begin{align*}
  \underline{\nabla}\cdot \underline{\underline{\tau }} &= \pdev{}{x_j}\left[ -P_S\delta_{ij} +2\mu \left( S_{ij} -\frac{1}{3} S_{kk}\delta_{ij} \right) +\mu_v S_{kk}\delta_{ij} \right] \\
                                                        &=-\underline{\nabla}P_S + \pdev{}{x_j} \left( 2\mu S_{ij}\right) +\pdev{}{x_j}\left( \mu _v-\frac{2}{3}\mu \right) S_{kk}\delta_{ij} 
\end{align*}
Prendendo in considerazione il secono termine:
\[
2\mu \pdev{}{x_j}\left[ \frac{1}{2}\left( \pdev{u_i}{x_j}+\pdev{u_j}{x_i}\right) \right]  = \mu \pdev{^2u_i}{x_{j}^2} + \mu \pdev{}{x_i}\pdev{u_j}{x_j}
\]
Riprendiamo l'equazione completa
\[
\underline{\nabla}\cdot \underline{\underline{\tau }} = -\underline{\nabla}\cdot P_S + \mu \nabla ^2 \underline{u} + \mu \underline{\nabla}\left( \underline{\nabla}\cdot \underline{u}\right)  + \left( \mu _v -\frac{2}{3}\mu \right) \underline{\nabla}\left( \underline{\nabla}\cdot \underline{u}\right) 
\]
\subsection{Navier Stokes Generali}
Abbiamo così raggiunto le equazioni di Navier Stokes Generali
\begin{equation}
  \label{eq:navi_stok_gen}
  \pdev{}{t}\left( \rho \underline{u}\right)  + \underline{\nabla}\cdot \left( \underline{u} \otimes \rho \underline{u}\right)  = -\underline{\nabla}P_S +\rho \underline{g}+\mu \nabla^2 \underline{u}+ \left( \mu _v+\frac{1}{3}\mu \right) \underline{\nabla}\left( \underline{\nabla}\cdot \underline{u}\right) 
\end{equation}
Posso sciogliere il primo termine $ \pdev{}{t}\left( \rho u_i\right)  $ in
\[
  \rho \pdev{u_i}{t}+ u_i \pdev{\rho }{t}  
\]
e il secondo $ \pdev{}{t}\left( u_i\rho u_j \right) $ in
\[
u_i\pdev{}{x_j}\left( \rho u_j\right) +\rho u_j\pdev{u_i}{x_j}
\]
Il secondo termine della prima equazione e il primo della seconda sommati indicano l' equazione della conservazione della massa moltiplicata per $ u_i $.
Dunque
\[
  u_i \left( \pdev{\rho }{t}+\pdev{}{x_j}\left( \rho u_j\right) \right) \rightarrow u_i\left( \pdev{\rho }{t}+\underline{\nabla}\cdot \left( \rho \underline{u}\right) \right) =0
\]
Arrivo così alla equazione di Navier-Stokes completa vettoriale
\begin{equation}
  \label{eq:NV_comp_vett}
  \rho \frac{ D \underline{u}}{t} = - \underline{\nabla}P_S +\rho \underline{g}+\mu \nabla ^2 \underline{u} + \left( \mu _v +\frac{1}{3} \mu \right) \underline{\nabla}\left( \underline{\nabla}\cdot \underline{u}\right) 
\end{equation}
Per il flusso incomprimibile
\[
\rho \frac{D \underline{u}}{Dt} = - \underline{\nabla} P + \rho \underline{g}+\mu \nabla^2 \underline{u}
\]
Moltiplico entrambi i membri per $ \frac{1}{\rho } $ e sostituisco $ \frac{\mu}{\rho } = \nu  $
\begin{equation}
  \label{eq:NV_flux_incompr}
  \frac{D\underline{u}}{Dt} = -\frac{1}{\rho }\underline{\nabla} P + \underline{g} + \nu  \nabla ^2 \underline{u}
\end{equation}


\subsubsection{Eulero}
Se non considero viscosità, attrito e considero il mio fluido un fluido ideale trovo un'equazione più semplice, chiamata equazione di \emph{eulero}
\begin{equation}
  \pdev{}{t}\left( \rho \underline{u}\right)  + \underline{\nabla}\cdot \left( \underline{u} \otimes \rho \underline{u}\right) = - \underline{\nabla} P_S +\rho \underline{g}
\end{equation}
Sostanzialmente l' equazione di eulero vale per numeri di Reynolds molto alti, dove l'attrito è trascurabile. Riprendendo \cref{eq:NV_flux_incompr} e considerando $ \nu =0 $ 
\begin{equation}
  \frac{D \underline{u}}{Dt}=\frac{1}{\rho }\underline{\nabla}P +\underline{g}
\end{equation}
Questa equazione è apparentemente più docile ma va considerato che la derivata sostanziale implica una non linearità dell'equazione.
A causa di ciò non è detto che la combinazione lineare di due soluzioni sia a sua volta soluzione.
%%% Local Variables:
%%% mode: latex
%%% TeX-master: "master"
%%% End:
