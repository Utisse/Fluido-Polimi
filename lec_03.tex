\lecture{3}{lun 27 feb 2023 14:30}{Tensione Superficiale, Capillarità, Cinematica}

\subsection{Tensione superficiale}
Il principio è lo stesso di una membrana solida costituita dalle forze delle molecole esterne.
Viene misurata in $ \unit{N/m} $. Studio qual è il legame tra tensione superficiale e il salto di pressione $ \Delta P=P_1 -P_2 $ sopportabile.
\begin{figure}[H]
  \centering
  \incfig{membrana}{}
  \caption{}
  \label{fig:membrana}
\end{figure}
Considerato $ P_b >P_a $, possiamo dire che $ \Delta\sigma =0$ perchè il problema è speculare, simmetrico rispetto a $ y $.
\begin{equation*}
  \label{eq:young_laplace_calc}
  \int_{-\Delta \theta}^{\Delta \theta} (P_b-P_a) \cos(\Delta \theta  ) 
\end{equation*}
\begin{equation}
  \label{eq:young_laplace}
 P_b-P_a = \frac{\sigma}{R} 
\end{equation}
L'\cref{eq:young_laplace} è valida in 2D. Nel 3D $ R $ è definita come $ \frac{1}{R}=\frac{1}{R_y}+\frac{1}{R_x} $\\

\subsection{Capillarità}

\begin{figure}[H]
  \centering
  \incfig{angolo_tensione}{}
  \caption{L'angolo $ \theta  $ che si forma a causa delle interazioni liquido-gas-solido è noto per sostanza. Inoltre $ \gamma_{12}=\gamma_{31}+\gamma_{23}\cos(\theta) $}
  \label{fig:angolo_tensione}
\end{figure}
\begin{figure}[H]
  \centering
  \incfig{tubocapillare_acqua}{}
  \caption{Tubo capillare pieno di acqua. $ H $ è la distanza verticale tra pelo libero e fondo del menisco. L'angolo che si forma è lo stesso (si veda \cref{fig:angolo_tensione}).}
  \label{fig:TuboCapillare_acqua}
\end{figure}
\begin{equation*}
  (P_a - P_1)\pi r^2-\rho g\pi r^2H=0
\end{equation*}
Il mio obiettivo è trovare il valore di $ H $, ma ancora non conosco $ P_1 $.\\
Trovo $ P_1 $:
\begin{eqnarray*}
  \int_S(P_1-P_a)\underline{n}ds &+ \int_{2\pi r}\underline{\sigma}ds=0\\
  \to P_1\pi r^2 \underline{e}z\,&\,\to 2\pi r\sigma \sin \theta  \underline{e}z\\
  (P_1-P_a)\bcancel{\pi }r^{\bcancel{2}}\bcancel{\underline{e_z}} &+ 2\bcancel{\pi} \bcancel{r} \sigma \sin  \theta \bcancel{\underline{e_z}}\\
  (P_1-P_a)r&+2\sigma \sin \theta 
\end{eqnarray*}
A questo punto posso trovare l'equazione per $ H $:
\begin{equation}
  H = \frac{2\sigma \sin \theta }{\rho gr}
\end{equation}
Chiaramente il discorso è diverso se si parla di acqua o di un altro liquido, come il mercurio. In particolare:
\twomini{
\begin{figure}[H]
  \centering
  \incfig{tubocapillare_acqua2}{}
  \label{fig:TuboCapillare_acqua2}
\end{figure}
}{
  \begin{equation*}
    \sin \theta >0 \quad P_1<P_{atm} \quad H>0
  \end{equation*}
}
Mentre per il mercurio
\twomini{
\begin{figure}
    \centering
    \incfig{tubocapillare_mercurio}{}
    \label{fig:tubocapillare_mercurio}
\end{figure}
}{
  \begin{equation*}
    \sin \theta <0\quad P_1>P_{atm} \quad H<0
  \end{equation*}
}
\section{Cinematica}
\twominisw{
  \begin{flushleft}
    \begin{figure}
      \incfig{descrizione_lagrangiana}{}
      \caption{Descrizione Lagrangiana: analizzo un volumetto di fluido $ \underline{r} $ dipende da $ (\underline{r_0};t_0 )$ e da $ t $}
      \label{fig:descrizione_lagrangiana}
  \end{figure}
\end{flushleft}
}{
  \begin{flushright}
    \begin{figure}
    \incfig{descrizione_euleriana}{}
    \caption{Descrizione Euleriana: Studio il campo vettoriale. Le proprietà come la velocità dipendono dalla posizione e dal tempo. Per esempio la velocità $ \underline{u}(\underline{x},t )$}
    \label{fig:descrizione_euleriana}
    \end{figure}
\end{flushright}
}{.45}{.45}
Questi due sistemi descrivono allo stesso moto e dunque:
\begin{equation*}
  \underline{u}((\underline{r_0}; t_0); t) = \underline{u}(\underline{x},t)
\end{equation*}
O più in generale:
\begin{equation}
  f(\underline{r},t) = f(\underline{x},t)
\end{equation}
Ora derivo una qualsiasi funzione $ f $.\\
Nella descrizione Lagrangiana:
\begin{equation}
  \frac{d}{dt}f[\underline{r}(r_0,t),t] = \frac{\partial f}{\partial r_1} \frac{\partial r_1}{\partial t} + \frac{\partial f}{\partial r_2} (\partial \frac{r_2}{\partial t} + \frac{\partial f}{\partial r_3} (\partial \frac{r_3}{\partial t} + \frac{\partial f}{\partial t} 
\end{equation}
Nella descrizione Euleriana:
\begin{equation}
  \frac{d}{dt}f=\pdev{f}{x_1}\pdev{x_1}{t} + \pdev{f}{x_2}\pdev{x_2}{t} + \pdev{f}{x_3}\pdev{x_3}{t}+\pdev{f}{t}
\end{equation}
Queste due forme sono interscambiabili. Dato che $ \pdev{x_i}{t} = u_i $ allora:
\begin{equation*}
  = \pdev{f}{x_1} u_1+\pdev{f}{x_2}u_2+\pdev{f}{x_3}u_3+\pdev{f}{t}
\end{equation*}
Dato che $(\pdev{f}{x_1} \, \pdev{f}{x_2} \, \pdev{f}{x_3}) = \underline{\nabla}f$ posso riscrivere la derivata come:
\begin{equation}
  \label{eq:derivata_sostanziale}
  \frac{Df}{Dt}=\underline{u}\cdot\underline{\nabla}f + \pdev{f}{t}
\end{equation}
che è detta \emph{derivata sostanziale}.
Nel caso di flusso stazionario il termine $ \pdev{f}{t} $ si annulla ma il termine $ \underline{u}\cdot\underline{\nabla}f $ no!
%%% Local Variables:
%%% mode: latex
%%% TeX-master: "master"
%%% End:
