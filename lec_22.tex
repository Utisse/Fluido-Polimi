\lecture{22}{gio 11 mag 2023 11:48}{Equazione di Von Karman}

\[
  \frac{{\delta \left( x\right) }}{x} = \frac{1}{\sqrt{Re}}\rightarrow  \delta \left( x\right) = x \sqrt{{\frac{\nu}{ax^nx}}}=\sqrt{\frac{\nu}{a}}x^{\frac{{1-n}}{2}}
\]
\begin{equation*}
  \delta \left( x\right) 
  \begin{cases}
	  \text{Cresce } n<1 \quad \left( n=0\rightarrow  \text{ Blasius} \right)\\
    \text{Costante } n =1 \\
    \text{Decresce } n> 1  
  \end{cases}
\end{equation*}
Equazione di 
\begin{equation}
  f^{\prime\prime\prime}+\frac{{n+1}}{2}ff^{\prime\prime}-nf^{\prime2}+n=0
\end{equation}
Le condizioni al contorno sono le stesse dell'equazione di Blasius \cref{eq:blasius} 
\begin{equation*}
  \begin{cases}
    \eta =0:\,\, f = f^{\prime}=0\\
    \eta \to :\,\, f^{\prime}\to 1
  \end{cases}
\end{equation*}
Per $ n>0 $: un fluido si scontra con un diedro. L'angolo che che si forma è $ \pi \theta = 2\frac{\pi n}{n+1} $ e $ U =ax^n $  Infatti $ \delta \left( x\right)  $ sarà:
\[
  \delta \left( x\right)  = \sqrt{\frac{\nu}{a}}x^{\frac{1-n}{2}}
\]
\begin{equation*}
  \delta
  \begin{cases}
    \text{ Cresce } n<1\\
    \text{ Costante } n=1\\
    \text{ Decresce} n>1
  \end{cases}
\end{equation*}
Quando $ n<0  $
%% TODO: RECUPERA
Vediamo lo sforzo a parete $ t_w $ 
\[
	\tau _w = \mu  \left \frac{d u}{d y}\right|_{y=0} = \mu a x^{\frac{{3n-1}}{2}}f^{\prime\prime}_0 \sqrt{\frac{a}{\nu }}
\]
\begin{equation*}
  \large
  \begin{cases}
    n>\frac{1}{3}: \tau_w \text{ cresce}\\ 
    n=\frac{1}{3}: \tau_w \text{ costante}\\ 
    n>\frac{1}{3}: \tau_w \text{ decresce} 
  \end{cases}
\end{equation*}
Nella zona esterna:\\
Equazione di bernulli flusso esterno: $  $ 
\begin{gather*}
  
\end{gather*}

Nella zona interna, sicuramente valgono le equazioni di Prandlt:\\
Equazione strato limite a parete: $ u\pdev{u}{x}+v\pdev{u}{y}=-\frac{1}{\rho }\pdev{P}{x}+ \nu \pdev{^2P}{x^2}$
\\
Se  il gradiente di pressione è favorevole, il flusso è accelerato e lo strato limite tende a diminuire:\\
In caso di gradiente avverso:
\begin{equation*}
  \pdev{P}{s}\gtrless 0 \rightarrow 
  \begin{cases}
    \pdev{U}{s}\gtrless 0\\
    
  \end{cases}
\end{equation*}
C'è un limite al cambio di curvatura, quando la curva diventa molto covessa una curva di velocità diventa convessa.
C'è ricircolo e dunque delle velocità negative: il flusso si è \emph{separato}.
Il punto di inizio di separazione è quando $ \left( \pdev{u}{y}\right)_0 = 0  $.\\
La condizione di gradiente avverso

\section{Equazione di Von Karman}

\begin{equation*}
  \begin{cases}
    \pdev{u}{x}+\pdev{v}{y}=0\\
    u\pdev{u}{x}+v\pdev{u}{y}-\nu \pdev{^2u}{y^2}=U\frac{d U}{d x}
  \end{cases}
\end{equation*}
Integriamo analizzando in primis $ v\pdev{u}{y} $ 
\[
  \displaystyle{\int_{0}^{\infty }}v\pdev{u}{y}dy = \cancel{{\left[ v\left( u-U\right) \right] ^\infty _0}} - \displaystyle{\int_{0}^{\infty }}\left( u-U\right) \frac{d v}{d y}dy=
\]

\[
  =+\displaystyle{\int_{0}^{\infty }}\left( u-U\right) \frac{d u}{d x}dy
\]
\[
  \displaystyle{\int_{0}^{\infty }}\left[  u\pdev{u}{x}dy+\left( u-U\right) \pdev{u}{x}-U\frac{d U}{d x}\right] dy=
  \displaystyle{\int_{0}^{\infty }}\nu \pdev{^2u}{y^2}dy=\left \nu \pdev{u}{y}\right|_0^\infty  = - \nu \left( \pdev{u}{y}\right) _0=-\frac{1}{\rho }\tau _w 
\]
\[
  \frac{d }{d x}\displaystyle{\int_{0}^{\infty }}\left( u^2-Uu\right) dy + \frac{d U}{d x}\displaystyle{\int_{0}^{\infty }} \left( u-U\right) dy = -\frac{1}{\rho }\tau _w
\] 
\[
  \frac{d }{d x}\displaystyle{\int_{0}^{\infty }}U^2\left( \frac{u^2}{U^2}-\frac{u}{U}\right) dy + \frac{d U}{d x}U\displaystyle{\int_{0}^{\infty }}\left( \frac{u}{U}-1\right) dy = -\frac{1}{\rho}\tau _w
\]
\[
  -\frac{d }{d x}\left[ U^2\displaystyle{\int_{0}^{\infty }}\frac{u}{U}\left( 1-\frac{u}{U}\right) dy\right] -\frac{d U}{d x}U\displaystyle{\int_{0}^{\infty }}\left( 1-\frac{u}{U}\right) dy = -\frac{1}{\rho }\tau _w
\]
I termini sono diventati gli spessori:
\begin{equation}
  \label{eq:von_karman}
  \frac{d }{d x}\left( U^2\theta \right)  + \frac{d U}{d x}U\delta ^*=\frac{1}{\rho }\tau _w
\end{equation}
Abbiamo trovato l'equazione di \emph{Von Karman}. La risoluzione non è banale ma posso aggiungere relazioni costitutive e rendere la risoluzione più veloce e semplice rispetto alle equazioni di Prandlt. Le relazioni costitutive sono del tipo:
\[
  F\left( \theta ,\delta^*,\tau _w\right) =0
\]
Lo sforzo a parete $ \tau _w $ è una forza per area (Pascal).
Se la volessimo adimensionalizzare troviamo $ C_f=\frac{\tau_w}{\frac{1}{2}\rho U^2} $.
Introduciamo anche il fattore di forma $ H =\frac{{\delta^*}}{\theta } $ (aiuta a capire se lo strato limite si separa).
Se introduco questi parametri all' interno dell'equazione di Von Karman \cref{eq:von_karman} trovo:
\begin{equation}
  \frac{d \theta }{d x}= \frac{C_f}{2}-\frac{\theta}{U}\frac{d U}{d x}\left( 2+H\right) 
\end{equation}
questa equazione è molto più semplice da trattare.

%%% Local Variables:
%%% mode: latex
%%% TeX-master: "master"
%%% End:
