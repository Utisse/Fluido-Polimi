\lecture{1}{mar 21 feb 2023 14:30}{intro}

Che cosa significa fluido? Quali sono le differenze con un solido?\\
I fluidi possono essere sia liquidi che gas.
Entrambi rientrano nella classificazione di Fluidi perchè obbediscono alla stessa (vasta) gamma di equazioni.
\begin{figure}[H]
    \centering
    \incfig{solliqgas}{}
    % \caption{}
    % \label{fig:cubo}
  \end{figure}
  Quando due oggetti solidi si toccano si ha un interazione tra i i protoni e gli elettroni.
  Lo stesso accade con i fluidi.
  Prendo in considerazione due molecole di un fluido ad una certa distanza $d$.
% \begin{figure}[H]
%     \centering
%     \incfig{molecole_fluido}
%   \end{figure}
  \twomini{
  \begin{figure}[H]
    \centering
    \incfig{dist_min}{}
  \end{figure}}{
  \begin{tabular}{|c|c|c|c|}
     & $F$ & $\lambda$ & Posizione\\
    \hline
    Solidi & grandi & $<d_0$ & ordinata \\
    Liquidi & grandi & $\approx d_0$ & no\\
    Gas & deboli & $\gg d_0$ & no
    \end{tabular}
  }
  Se le molecole sono lontane è valida l'approssimazione di gas perfetto $P=\rho RT$.\\
  \section{Ipotesi del continuo}
  Se considero un volume molto piccolo noto come la densità ($\rho$) varia molto rapidamente.
  Mano a mano che aumento la dimensione del mio volumetto la densità si stabilizza fino a divenire costante.
  Su questo fatto sperimentale si basa la validità dell'ipotesi del continuo.\\
  Grazie ad essa posso non considerare il fluido al livello molecolare ma parlare di fluido in toto, con alcune caratteristiche generali.\\
  % \begin{figure}[H]
    % \centering
    % \incfig{ip_continuo}
    % \end{figure}
  Tutto questo ovviamente dipende da $ \lambda  $ ovvero il cammino libero medio.
  Nell'aria $ \lambda \approx 70 \unit{nm} $, mentre nell'acqua è $ \lambda \approx 0.1 \unit{nm}$\\
  \subsection{Numero di Knudsen}
  Introduciamo inoltre una quantità adimensionale chiamata numero di Knudsen.
  \begin{equation}
    \label{eq:knudsen}
    k = \frac{\lambda}{L}
  \end{equation}
  Il numero di Knudsen mette in relazione il cammino libero medio ($ \lambda  $) e la scala di riferimento ($ L $), che è la misura che caratterizza al meglio il nostro problema (per un flusso intorno ad un ala è la corda di quest'ultima, per un tubo il suo diametro, ecc.).\\
  Analizzando il numero di Knudsen si può valutare la validità dell'ipotesi del continuo:
  \begin{equation}
    \label{eq:cont_raref}
    \begin{aligned}
      \text{Continuo} \to\, &k\ll 1\\
      \text{Rarefatto} \to\, &k\gg 1
    \end{aligned}
  \end{equation}
  Si noti inoltre che se il fluido è rarefatto influisce anche su altri fenomeni fisici come la diffusione termica.\\
  Consideriamo ora un fluido con regioni di temperatura diverse (stratificato).
  \twomini{
  \begin{figure}[H]
    \centering
    \incfig{fluxDiffTemp}{}
    \end{figure}
  }{
    \centering
    Legge di Fourier:
    \begin{equation}
      \label{eq:fourier}
      \underline{q} = -k\underline{\nabla}T \qquad [\unit{J/m^2s}] 
    \end{equation}
  }
  La temperatura nel serbatoio tende a uniformarsi.
  La legge di Fourier (\cref{eq:fourier}) definisce il flusso di calore che omogenizza la temperatura. La stessa cosa avviene per la quantità di moto.
  \twomini{
  \begin{figure}[H]
    \centering
    \incfig{diffQM_Graph}{.5}
    \end{figure}
  }{
    \centering
    Viscosità dinamica:
    \begin{equation}
      \label{eq:visc_din}
      \mu  \quad [\unit{kg/ms}] 
    \end{equation}
    \begin{equation}
      \label{eq:sforzo}
      \tau =\mu \frac{d\underline{u}}{d y}
    \end{equation}
    \begin{equation}
      \label{eq:visc_cin}
      v =\frac{\mu}{\rho} \quad [\unit{m^2 / s}]
    \end{equation}
  }
  La velocità tende a omogenizzarsi: lo strato più lento accelera mentre lo strato più veloce rallenta sbattendo contro lo strato più lento.
  L'attrito tra i due strati si chiama sforzo e \cref{eq:sforzo} che la definisce è valida solo per fliudi newtoniani: negli altri fluidi lo sforzo non dipende esplicitamente da $ \frac{d \underline{u}}{dy} $. Im questo caso il fluido è detto \emph{viscoelastico}.
  $ \mu  $ è detta \emph{viscosità dinamica} (\cref{eq:visc_din}) e $ v $ è detta \emph{viscosità cinematica} (\cref{eq:visc_cin}).\\
  In generale la viscosità dipende dalla temperatura:
  \begin{itemize}
  \item Nei gas la viscosità aumenta con la temperatura $ \to T\uparrow \mu\uparrow$ (con $ T $ elevato ad una costante $ k $    
  \item Nei liquidi la viscosità diminuisce all'aumentare della temperatura $ \to T\uparrow \mu\downarrow  $ (rompo legami intermolecolari con energia termica) 
  \end{itemize}
  
%%% Local Variables:
%%% mode: latex
%%% TeX-master: "master"
%%% End: