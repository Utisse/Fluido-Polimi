\lecture{21}{mar 09 mag 2023 14:30}{Strato Limite su lamina piana semi-infinita}

\[
  \delta =\delta \text{ tale che } u\left( x,\delta_{99}\right)  = 0.99 U_e
\]
Spessore di spostamento.\\
Nella situazione ideale la velocità rimane costante fino al raggiungimento della superficie.
Nel caso reale questo non succede.
Possiamo interpretare questa semplificazione come un difetto di portata.
\begin{figure}[ht]
    \centering
    \incfig{corrente_spostata}{}
    % \caption{corrente_spostata}
    \label{fig:corrente_spostata}
\end{figure}
\[
  \displaystyle{\int_{0}^{\infty }}\left[ U_e - u \left( y\right) \right] dy=\displaystyle{\int_{0}^{\delta ^*}}U_edy = U_e \delta ^*
\]

\[
  \delta ^* = \displaystyle{\int_{0}^{\infty }}\left( 1- \frac{u}{U_e}\right) dy
\]
Spessore di spostamento: $ \delta ^*<\delta _\alpha  $ \\
Prima che il flusso interagisca con la parete il profilo di velocità sarà indisturbato, $ U_e $.
Quando poi il flusso incontra la superficie lo strato limite crescerà (se superficie semplicemente piana).
Se considero una linea di corrente qualsiasi (per forza deviata verso l'alto).
Prendo in considerazione un punto $ x $, in quel punto la distanza verticale tra la linea di corrente e la superficie sarà $ h $.
Sul punto precedente, quando la linea non era disturbata, la distanza verticale sarà $ h-\delta ^+ $
Lo sforzo sulla superficie $ \tau =\tau _w $.
\[
  \underbrace{{\displaystyle{\int_{0}^{h}}\rho u^2dy}}_{\text{Out}} - \underbrace{{\displaystyle{\int_{0}^{h-\delta ^*}}\rho U_e^2dy}}_{\text{In}}{ = - \displaystyle{\int_{0}^{x}}\tau _wdx = -\rho U_e^2\theta 
\]
\[
  \rho U^2_e \theta - \left[ \displaystyle{\int_{0}^{h}}\rho \left( u^2-U_e^2\right) dy - \rho U_e^2\left( -\delta ^*\right) \right] 
\]
\[
  \theta = \left[ \displaystyle{\int_{0}^{h}}\left( \frac{u^2}{U_e^2}-1\right) dy + \displaystyle{\int_{0}^{h}}\left( 1-\frac{u}{U_e}\right) dy \right]  = -\displaystyle{\int_{0}^{h}}\left( \frac{u^2}{U_e^2}-\frac{u}{U_e}\right) dy =
\]
\begin{equation}
   =\displaystyle{\int_{0}^{h}}\frac{u}{U_e}\left( 1-\frac{u}{U_e}\right) dy
\end{equation}
\emph{(il rapporto tra $ \delta^*  $ e $ \theta  $ si chiama fattore di forma.)}\\
\hline
\vspace{1ex}
\subsection{Strato limite su lamina piana semi-infinita}
Studiamo lo strato limite (anche ripetibile in laboratorio con \textit{trucchi}). Lo strato limite si chiama di \emph{Blasius}. La corrente che arriva verrà frenata. Più sono vicino alla parete più la corrente verrà rallentata. Le linee di corrente vengono lentamente deflesse e lo strato limite aumenta il suo spessore. Avvicinandoci alla parete il difetto di portata aumenta. Lo spessore dello strato aumenta progressivamente. Il flusso è uniforme, pressione e velocità uniforme, poi viene disturbato dalla lamina. Dato che la lamina è semi-infinita dobbiamo considerare una scala locale, dunque il Reynolds sarà locale e cambierà in base alla distanza $ x $ chiamato $ Re_x $
\[
  Re_x = \frac{Ux}{\nu }
\] 

\begin{equation*}
  \large
  \begin{cases}
    \pdev{u}{x}+\pdev{v}{y}=0\\
    u\pdev{u}{x}+v\pdev{u}{y}=U_e \cancel{\frac{d U_e}{d x}} + \nu \pdev{^2u}{y^2}
  \end{cases}
\end{equation*}
Rimane solo la componente dissipativa. Mancano le condizioni al contorno
\begin{equation*}
  \begin{cases}
    u\left( x,0\right)  = v\left( x,0\right) =0\\
    u\left( x,y\right)  \rightarrow}_{y\to\infty }U_e\\
    u\left( 0,y\right)  = U_e
  \end{cases}
\end{equation*}
Da ora in poi $ U_e =U $, corrente asintotica. 
Nell'origine la prima e la terza condizione si contraddicono.
Dunque ho bisogno di una soluzione trovata in un altro modo nell'origine.
Possiamo dire che $ \delta \left( 0\right) =0 $ oppure risolvere Navier-Stokes nell' origine.\\
Se usiamo unità dimensionali $ h $ varia con $ x $. Se invece scalo tutto con $ \frac{y}{\delta } $ tutto collassa in una curva detta \emph{autosimile}.
\[
  \underline{\nabla}\cdot \underline{u} = 0 \rightarrow U = \pdev{\psi}{y};\,\,v = -\pdev{\psi}{x}
\]
Non posso definire il potenziale perchè $ \nabla\times \underline{u} \neq 0 \rightarrow \cancel{\exists} \phi  $\\
Definisco $ \eta $: 
\[
  \rightarrow \eta = \frac{y}{\delta \left( x\right)} \quad \frac{u\left(  x,y\right) }{U}=h\left(\eta\right) 
\]
Definisco $ \psi $ 
\[
  \psi = U\delta \left( x\right) f\left( \eta\right) \quad \text{con } f\left( \eta\right)  = \displaystyle{\int_{0}^{\eta}}f\left( \eta^{\prime}\right) d\eta^{\prime}
\]
infatti 
\[
  \pdev{\psi}{y} = U\delta \frac{d f}{d \eta}\pdev{\eta}{y}=U \cancel{\delta} h \cancel{\frac{1}{\delta }} = U h =u
\]

\[
  v = -\pdev{\psi}{x}= -U \frac{d \delta }{d x}f\left( \eta\right) -U \delta  \frac{d f}{d \eta}\left( -\frac{y}{\delta }\frac{d \delta }{d x}\right)  = -U \frac{d \delta }{d x}\left[ f-\eta\frac{d f}{d \eta}
\]

\begin{gather*}
    U \frac{d f}{d \eta}\pdev{}{x}\left( U\frac{d f}{d \eta}\right) +U \frac{d \delta }{d dx}\left[ -f +\eta\frac{d f}{d \eta}\right] \frac{d }{d \eta}\left( U\frac{d f}{d \eta}\pdev{\eta}{y }\\
      \nu \frac{d }{d \eta}\left[ \frac{d }{d \eta}\left( \frac{d f}{d \eta}\right) \pdev{\eta}{y}\right] \pdev{\eta}{y}
\end{gather*}
Passo alla seconda equazione di Prandlt
\begin{gather*}
  U^2\frac{d f}{d \eta} \frac{d }{d \eta }\left( \frac{d f}{d \eta }\right) \frac{d \eta }{d x}+ U^2\pdev{\delta }{x}\left[ -f+\eta \frac{d f}{d \eta }\right] \frac{d^2f }{d\eta ^2 } \frac{1}{\delta }=\\
  =\nu U\frac{d }{d \eta }\left( \frac{d ^2f}{d \eta ^2}\cdot \frac{1}{\delta }\right) \cdot \frac{1}{\delta }
\end{gather*}
Definisco $ \frac{d f}{d \eta }=f^\prime  $ e continuo:
\begin{gather*}
  U^2f^{\prime} f^{\prime\prime} \left(  - \frac{y}{\delta }\right) \frac{d \delta }{d x}+U^2\frac{d \delta }{d x}\left[ -f \eta f^'\right] f^{\prime\prime}\frac{1}{\delta }=\nu Uf^{\prime\prime\prime}\frac{1}{\delta ^2}\\
  \cancel{{-U^2 f^{\prime}f^{\prime\prime}\frac{\eta}{\delta }\delta^{\prime}}} - U^2ff^{\prime\prime}\delta^{\prime} \frac{1}{\delta }+{U^2\eta f^{\prime}f^{\prime\prime}\frac{1}{\delta }\delta^{\prime} = \nu Uf^{\prime\prime\prime} \frac{1}{\delta ^2}
\end{gather*}

\[
  -ff^{\prime\prime} \delta \delta^{\prime}} = \frac{\nu}{U}f^{\prime\prime\prime}
\]
Se impongo $ C\delta \delta^{\prime} = \frac{\nu}{U} \rightarrow \frac{1}{2}C\delta ^2=\frac{\nu}{U}x+D$. Chiamo in causa le condizioni al contorno.
\[
  \delta \left( 0\right) =0 \rightarrow D = 0
\]
trovo
\[
  \delta ^*=\frac{{2\nu x}}{CU}
\]
Per $ C = 2 $ si ottiene
\[
  \delta = \sqrt{\frac{{\nu x}}{U}} \qquad \frac{\delta}{x} = \frac{1}{\sqrt{Re_x}}
\]
Sostituisco e trovo
\[
  f^{\prime\prime\prime} = -\frac{U}{\nu }\delta \delta^{\prime} ff^{\prime\prime} =- \frac{1}{2}f f^{\prime}
\]
E trovo l'eqauazione di Blasius, un'equazione alle derivate ordinarie, la soluzione è il probfilo di velocità autosimile.
\begin{equation}
  \label{eq:blasius}
  f^{\prime\prime\prime} + \frac{1}{2}f f^{\prime\prime} = 0
\end{equation}
Ricordiamo le condizioni al contorno:
\begin{gather*}
  y = 0 \rightarrow \eta =0\\
  \eta  = 0 \rightarrow f = f^{\prime}=0\\
  \eta \to \infty \rightarrow f^{\prime}\to\infty 
\end{gather*}
[[Tutto questo è mostrato a pagina 357 del Kundu]]

\begin{gather*}
  \delta _{99} \rightarrow \eta \approx   4.93\\
  \delta ^* \approx  1.72\, \delta \\
  \theta \approx 0.664\,\delta\\
  \tau _w =\mu \left\frac{d u}{d y}\right|_0 = \mu  U \frac{d }{d \eta }f^{\prime} \pdev{\eta }{y}=\eta U f^{\prime\prime} \frac{1}{\delta}  \approx 0.332 \rho U^2 \frac{1}{\sqrt{Re_x}}
\end{gather*}
%%% Local Variables:
%%% mode: latex
%%% TeX-master: "master"
%%% End:
